%\documentclass[10pt,a4paper]{article}

% headsepline: Linie am oberen Blattrand unterhalb der Seitennummer
% bibtotoc: Aufnahme des Literaturverzeichnisses ins Inhaltsverzeichnis
\documentclass[a4paper,headsepline,bibtotoc]{scrreprt}
\usepackage[utf8]{inputenc}
\usepackage{amsmath}
\usepackage{amsfonts}
\usepackage{amssymb}
% Einstellungen bez. des 'scrreprt'-Stils
% Caption Schriftstil und -Groesse
\renewcommand{\capfont}{\footnotesize}
\renewcommand{\caplabelfont}{\footnotesize\bfseries}
\typearea{15}  %Einstellung des Verh�ltnisses Gr��e des Textes zur Papiergr��e
% Sprache
\usepackage[german,english]{babel}
\selectlanguage{german}
\setlength{\parindent}{0pt}


\usepackage{pgfplots}
\usepackage{tikz}
\usepackage{caption}
\usepackage{subcaption}
\usepgfplotslibrary{external}
\usepackage{pdfpages}
\usepackage{lipsum}  




\addto\extrasgerman{\renewcommand{\figurename}{Abb.}}
\addto\extrasgerman{\renewcommand{\tablename}{Tab.}}

% Bilder
\usepackage[rflt]{floatflt}
\usepackage{epsfig,wrapfig}

% Programmablaufplaene (Struktogramme, Nassi-Schneidermann-Diagramme)
% Dieses Paket ist nicht standardm��ig im CIP-Pool installiert
% \usepackage{nassi}

% Mathematische Symbole
\usepackage{amsmath,amssymb}

% Tabellen
\usepackage{longtable,lscape}
\usepackage{multirow}
\usepackage{tabularx}

% Kopfzeilen
\usepackage{fancyheadings}
\pagestyle{plain}
\renewcommand{\chaptermark}[1]{\markboth{#1}{}}
\renewcommand{\sectionmark}[1]{\markboth{\thesection\ #1}{}}
\lhead[\fancyplain{}{\sl\leftmark}]%
      {\fancyplain{}{\sl\leftmark}}
\rhead[\fancyplain{}{\sl\thepage}]%
      {\fancyplain{}{\sl\thepage}}
\cfoot{}

% Aufgabenstellung
\usepackage{iagkopf}

\graphicspath{ {./images/} }

% Listenerscheinung
\setlength{\itemsep}{0ex}
\setlength{\parsep}{0ex}
\setlength{\parskip}{2mm}


\begin{document}
\sloppy

% Seitennumerierung bis zum Beginn der Einleitung auf kleine roemische Zahlen setzen
\pagenumbering{roman}

% neue Befehle
%\newcommand{\file}[1]{{\sffamily\slshape #1}}
\newcommand{\file}[1]{\mdseries\textsl{\textsf{#1}}}
\newcommand{\sbr}[1]{\texttt{#1}}
\newcommand{\var}[1]{\mdseries\textsl{\texttt{#1}}}
\newcommand{\cmd}[1]{\uppercase{\texttt{#1}}}

% Titelseite
\title{Titel der Abschlussarbeit}

\author{Bachelor/Master Thesis \\
        by \\
        Can. Aer./B.\ Sc.\ Vorname Nachname  }

\publishers{conducted at the \\
            Institute of Aerodynamics and Gas Dynamics \\
            University of Stuttgart.
            \\[5ex]
            Stuttgart, January 2023}

\date{}


\selectlanguage{english}

\maketitle

\newpage
\thispagestyle{empty} 
\section*{}
\newpage

% Aufgabenstellung einbinden
\addcontentsline{toc}{chapter}{Assignment}
\pagestyle{plain}
\includepdf[pages=1]{aufgabenstellung_rlindi_msc}


\newpage
\thispagestyle{empty} 
\section*{}
\newpage

% �bersicht
\newpage
\chapter*{Erkl\"arung}
Hiermit versichere ich, dass ich diese Masterarbeit selbstständig mit Unterstützung der Betreuer angefertigt und keine
anderen als die angegebenen Quellen und Hilfsmittel verwendet habe.
Die Arbeit oder wesentliche Bestandteile davon sind weder an dieser noch an einer anderen Bildungseinrichtung bereits zur Erlangung eines Abschlusses eingereicht worden.
Ich erkläre weiterhin, bei der Erstellung der Arbeit die einschlägigen Bestimmungen zum Urheberschutz fremder Beiträge entsprechend den Regeln guter wissenschaftlicher Praxis1 eingehalten zu haben. Soweit meine Arbeit fremde Beiträge (z.B. Bilder, Zeichnungen, Textpassagen etc.) enthält, habe ich diese Beiträge als solche gekennzeichnet (Zitat, Quellenangabe) und eventuell erforderlich gewordene Zustimmungen der Urheber zur Nutzung dieser Beiträge in meiner Arbeit eingeholt. Mir ist bekannt, dass ich im Falle einer schuldhaften Verletzung dieser Pflichten die daraus entstehenden Konsequenzen zu tragen habe. 
\\
\\
\\

………………………………………\\
Ort, Datum, Unterschrift



\newpage
\thispagestyle{empty} 
\section*{}
\newpage

\addcontentsline{toc}{chapter}{Abstract}
\include{abstract}

\newpage
\thispagestyle{empty} 
\section*{}
\newpage

\addcontentsline{toc}{chapter}{Kurzfassung}
\newpage
\chapter*{Kurzfassung}

Abstract deutsch\\
\lipsum[1-2]




% Nomenklatur
\addcontentsline{toc}{chapter}{Nomenclature}
\include{nomenklatur}

% Neue Kapitel starten auf ungerader Seitenzahl (rechte Seite Buch)
\newpage
\thispagestyle{empty} 
\section*{}
\newpage

% Abkuerzung
\addcontentsline{toc}{chapter}{Abbreviations}
\newpage
\chapter*{Abbreviations}
\begin{tabular}{ll}
  \vspace{1mm}
  DDPG     & Deep Deterministic Policy Gradient\\
  \vspace{1mm}
  DG      & Discontinuous Galerkin\\
  \vspace{1mm}
  DGSEM      & Discontinuous Galerkin Spectral Element Method \\ 
  \vspace{1mm} 
  DOF     & Degree of Freedom\\
  \vspace{1mm}
  EBW5     & Elementwise Blending of the Weak DG Formulation with $N=5$\\
  \vspace{1mm}
  EBS5     & Elementwise Blending of the Split DG Formulation with $N=5$\\
  \vspace{1mm}
  FV      & Finite volume\\
  \vspace{1mm}  
  LG      & Legendre Gauss\\  
  \vspace{1mm}
  LGL      & Legendre Gauss Lobatto\\
  \vspace{1mm}
  LLF     & Local Lax-Friedrichs\\
  \vspace{1mm}
  MC     & Monte Carlo\\
  \vspace{1mm}
  MDP     & Markov Decision Process\\
  \vspace{1mm}
  MLP     & Mulit Layer Perceptron\\
  \vspace{1mm}
  PDE      & Partial Differential Equation\\
  \vspace{1mm}
  RK     & Runge Kutta\\
  \vspace{1mm}
  RL      & Reinforcement Learning\\
  \vspace{1mm}
  SAC     & Soft Actor Critic\\  
  \vspace{1mm}
  SBP     & Summation By Parts\\
  \vspace{1mm}
  SEB5     & Subelementwise Blending with $N=5$\\
  \vspace{1mm}
  SEB9     &  Subelementwise Blending with $N=9$\\
  \vspace{1mm}
  SL     & Supervised Learning\\
  \vspace{1mm}
  TD     & Temporal Difference\\
  \vspace{1mm}
  TD3     & Twin Delayed Deep Deterministic Policy Gradient\\ 
  \vspace{1mm}
  TVB      & Total Variation Bounded\\
  \vspace{1mm}
  WENO      & Weighted Essentially Non-Oscillatory\\
\end{tabular}


\newpage
\thispagestyle{empty} 
\section*{}
\newpage

% Inhaltsverzeichnis
\addcontentsline{toc}{chapter}{Table of Contents}
\tableofcontents




\pagestyle{plain}
\renewcommand{\chaptermark}[1]{\markboth{#1}{}}
\renewcommand{\sectionmark}[1]{\markboth{\thesection\ #1}{}}
\lhead[\fancyplain{}{\sl\leftmark}]%
      {\fancyplain{}{\sl\leftmark}}
\rhead[\fancyplain{}{\sl\thepage}]%
      {\fancyplain{}{\sl\thepage}}
\cfoot{}


\newpage
\thispagestyle{empty} 
\section*{}
\newpage

\newpage
\thispagestyle{empty} 
\section*{}
\newpage

% Seitennumerierung ab der folgenden Einleitung auf arabische Zahlen setzen
\pagenumbering{arabic}



% Einleitung
\newpage
\chapter{Introduction}
\lipsum
\cite{gibbs1899}
\cite{vonneumann1950}
\cite{jameson1981}

% Veruchsanordnung bzw. verwendete Software ******************

% Text
\newpage
\chapter{Experimental setup}
\lipsum
\section{Navier-Stokes Equation}
\lipsum[1]
\subsection{Euler Equation}
\lipsum[2]


% Ergebnisse ******************
\newpage

\chapter{Results and Validation}
\lipsum[1]

\section{Test Cases}\label{test cases}
\paragraph*{test case 1} 
\lipsum[2]


% Zusammenfassung ******************

% Text


\newpage
\chapter{Conclusion and Outlook}
\section{Conclusion}
\lipsum

\section{Outlook}
\lipsum









% Literatur
\bibliographystyle{literaturstil}
\bibliography{literatur}

\clearpage
% Anhang ******************

%Text
\addcontentsline{toc}{chapter}{Appendix}
\include{anhang}


\end{document}